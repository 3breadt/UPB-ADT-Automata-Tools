\subsection{Konvertierung zu Regulärem Ausdruck}

Es gibt zahlreiche Methoden um Endliche Automaten zu Regulären Ausdrücken zu konvertieren,
welche unterschiedliche Ergebnisse liefern. Die Konvertierung von Endlichem Automaten zu
Regulärem Ausdruck ist nicht eindeutig.

In diesem Projekt wurde die Algebraische Methode nach Brzozowski implementiert.

\subsubsection{Algebraische Methode}

Bei der algebraischen Methode wird der Automat als Gleichungssystem aus Regulären 
Ausdrücken betrachtet.

Für alle ausgehenden Übergänge eines Zustands $Q_i$ wird eine Gleichung aufgestellt.
Sei $a$ das Eingabesymbol des Übergangs und $Q_j$ der Zielzustand der Transition. Dann 
lautet der Reguläre Ausdruck für diese Transition $a.Q_j$.

Für jeden Zustand $Q_i$ werden dann die Regulären Ausdrücke der ausgehenden Transitionen
mit einer Oder-Verknüpfung zusammengefasst, z.B. $Q_0 = a.Q_1 | b.Q_2 | c.Q_3$.

Es ergibt sich ein Gleichungssystem aus Regulären Ausdrücken für jeden Zustand.

Der Reguläre Ausdruck, welcher die Sprache des Endlichen Automaten beschreibt, ergibt
sich durch Lösung des Gleichungssystem nach dem Startzustand.

\subsubsection{Lösungsreihenfolge nach Brzozowski}

Je nach Lösungsreihenfolge des Gleichungssystem ergibt sich ein unterschiedlicher
Regulärer Ausdruck.

Die algebraische Methode nach Brzozowski beschreibt eine solche Lösungsreihenfolge. Der
folgende Pseudocode beschreibt die Lösung des Gleichungssystems nach Brzozwski.

%Workaround um Probleme mit UTF-8 und listings zu beheben
\lstset{language=C, basicstyle=\footnotesize}
\lstset{literate=%
{Ö}{{\"O}}1
{Ä}{{\"A}}1
{Ü}{{\"U}}1
{ß}{{\ss}}1
{ü}{{\"u}}1
{ä}{{\"a}}1
{ö}{{\"o}}1
{ε}{{$\epsilon$}}1
{Σ}{{$\Sigma$}}1
}
\lstset{float,frame=lines,language=C,
caption={Pseudocode für die algebraische Methode nach Brzozowski.},
label=lst:Jaeger-checkDistance}
\begin{lstlisting}
Sei m die Anzahl der Zustände.
Sei Q der Vektor der Größe m der alle Zustände enthält.
Sei Q[0] der Startzustand.
Sei Σ das Eingabealphabet.
Sei B ein Vektor der Größe m. In ihm werden die Regulären Ausdrücke
  für jeden Zustand gespeichert
Sei A eine Matrix der Dimension m x m. In ihr werden alle Transitionen
  gespeichert.
Sei e der Reguläre Ausdruck, welcher die Sprache des Automaten beschreibt.
  
Für i = 0 bis m
  Wenn Q[i] Endzustand ist
    B[i] := ε
  Sonst
    B[i] := NULL
	
  Für j = 0 bis m
    Für jedes Eingabesymbol a aus Σ
      Wenn ein Übergang von Q[0] nach Q[1] mit a existiert
        a[i,j] := a[i,j] | a
	
Für n = m-1 bis 0
  B[n] = A[n,n]* . B[n]
  Für j = 1 bis n
    A[n,j] := A[n,n]* . A[n,j];
  Für i = 1 bis n
    B[i] := B[i] | A[i,n] . B[n]
    Für j = 1 bis n
      A[i,j] := A[i,j] | A[i,n] . A[n,j]
      
e = B[0]
\end{lstlisting}

