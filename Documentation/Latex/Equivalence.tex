\section{Äquivalenzprüfung}

Eine der Anforderungen an das Projekt war es, eine Methode zum Überprüfen der Äquivalenz 
zwischen verschiedenen Darstellungen einer Regulären Sprache zu finden.

Generell kann jede Darstellungsform in einen Endlichen Automaten überführt werden. Somit
war es nur notwendig eine Methode zu implementieren, welche die Äquivalenz zweier endlicher
Automaten überprüft.

Hierzu werden die beiden zu vergleichenden Automaten zunächst minimiert.

Laut Definition müssen minimale Automaten, welche die gleiche Sprache darstellen, die
gleiche Anzahl an Zuständen haben.

Falls dies gegeben wird, dann sind die beiden endlichen Automaten äquivalent, wenn die
beiden Startzustände äquivalent sind.

Äquivalenz von Zuständen ist dabei wie folgt definiert:

Seien $A$ und $B$ Zustände. 

Ist $A$ Endzustand und $B$ nicht (oder umgekehrt), so sind $A$ und $B$ nicht äquivalent.

Ansonsten sind $A$ und $B$ genau dann äquivalent, wenn für jedes
Eingabesymbol $e$ aus dem Eingabealphabet $\Sigma$, mit $\delta(A,e,P)$ und $\delta(B,e,Q)$,
gilt, dass $P$ und $Q$ äquivalent sind.
 
In der gewählten Implementation wird diese Definition rekursiv von den Starzuständen aus
angewandt und so die Äquivalenz der Automaten festgestellt.